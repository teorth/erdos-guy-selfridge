%\pdfoutput=1
\documentclass[12pt,a4paper,reqno]{amsart}
\newcommand\hmmax{0}
\newcommand\bmmax{0}
\usepackage{amssymb}
\usepackage{amscd}
\usepackage[pdftex,pdfpagelabels]{hyperref}
\usepackage{enumerate}
\usepackage{comment}
%\usepackage{psfig}
\usepackage{graphicx}
\usepackage{cleveref}
\usepackage{siunitx}
\usepackage{tikz-cd}
\usepackage{stix}
\usepackage{bm}
\DeclareMathAlphabet\mathbfcal{LS2}{stixcal}{b}{n}
\numberwithin{equation}{section}

%\usepackage{mathabx}


\usepackage{mathtools}%                  http://www.ctan.org/pkg/mathtools
\usepackage[tableposition=top]{caption}% http://www.ctan.org/pkg/caption
\usepackage{booktabs,dcolumn}%           http://www.ctan.org/pkg/dcolumn + http://www.ctan.org/pkg/booktabs

% Lighter notation.
%\newcommand*\mc[1]{\multicolumn{1}{c}{#1}}
%\newcommand*\tupref[2]{\href{http://math.mit.edu/~primegaps/tuples/admissible_#1_#2.txt}{\num{#2}}}



%\DeclareMathOperator*\Kl{Kl} (commented because yield bad display for  \Kl_q %replaced with \newcommand... )
%\DeclareMathOperator*\FT{FT} (commented because yield bad display for \FT_q
%replaced with \newcommand...)

\DeclareMathOperator*\swan{swan}
\DeclareMathOperator*\cond{cond}
\DeclareMathOperator*\Gal{Gal}

\newcommand{\FT}{\mathrm{FT}}
\newcommand{\Kl}{\mathcal{K}\ell}
% Setup for ``caption''.
%\DeclareCaptionLabelSeparator{separation}{:\quad}
%\captionsetup{
  %font=small,
  %labelfont=sc,
  %labelsep=separation,
  %width=0.8\textwidth
%}

\DeclareFontFamily{OT1}{rsfs}{}
\DeclareFontShape{OT1}{rsfs}{n}{it}{<-> rsfs10}{}
\DeclareMathAlphabet{\mathscr}{OT1}{rsfs}{n}{it}

\addtolength{\textwidth}{3 truecm}
\addtolength{\textheight}{1 truecm}
\setlength{\voffset}{-.6 truecm}
\setlength{\hoffset}{-1.3 truecm}

\theoremstyle{plain}

\newtheorem{theorem}{Theorem}[section]
%\newtheorem{theorem}[theorem]{Theorem}
\newtheorem{proposition}[theorem]{Proposition}
\newtheorem{lemma}[theorem]{Lemma}
\newtheorem{corollary}[theorem]{Corollary}
\newtheorem{conjecture}[theorem]{Conjecture}
\newtheorem{heuristic}[theorem]{Heuristic}
\newtheorem{principle}[theorem]{Principle}
\newtheorem{question}[theorem]{Question}
\newtheorem{problem}[theorem]{Problem}
\newtheorem{claim}[theorem]{Claim}

\theoremstyle{definition}

%\newtheorem{roughdef}[subsection]{Rough Definition}
\newtheorem{definition}[theorem]{Definition}
\newtheorem{remark}[theorem]{Remark}
\newtheorem{remarks}[theorem]{Remarks}
\newtheorem{example}[theorem]{Example}
\newtheorem{examples}[theorem]{Examples}
%\newtheorem{problem}[subsection]{Problem}
%\newtheorem{question}[subsection]{Question}

\renewcommand\P{\mathbb{P}}
\newcommand\E{\mathbb{E}}
\newcommand\Var{\mathrm{Var}}
\newcommand\R{\mathbb{R}}
\newcommand\Z{\mathbb{Z}}
\newcommand\F{\mathbf{F}}
\newcommand\N{\mathbb{N}}
\newcommand\n{\mathbf{n}}
\renewcommand\a{\mathbf{a}}
\renewcommand\b{\mathbf{b}}
\renewcommand\j{\mathbf{j}}
\renewcommand\k{\mathbf{k}}
\renewcommand\v{\mathbf{v}}
\renewcommand\t{\mathbf{t}}
\renewcommand\r{\mathbf{r}}
\renewcommand\l{\mathbf{l}}
\newcommand\X{\mathbf{X}}
\newcommand\T{\mathbf{T}}
\newcommand\Y{\mathbf{Y}}
\newcommand\A{\mathbf{A}}
\newcommand\W{\mathbf{W}}
\newcommand\C{\mathbb{C}}
\newcommand\Q{\mathbb{Q}}
\renewcommand\Re{{\operatorname{Re}}}
\renewcommand\Im{{\operatorname{Im}}}
\newcommand\Log{{\operatorname{Log}}}
\newcommand\lcm{{\operatorname{lcm}}}
\renewcommand\gcd{{\operatorname{gcd}}}
\newcommand\eps{\varepsilon}
\newcommand\deriv{\zeta}
\newcommand\zero{\lambda}

\renewcommand{\mod}{\bmod}

\parindent 0mm
\parskip   5mm


\begin{document}

\title{Notes on the Guy-Selfridge problem}

\author{Terence Tao}
\maketitle

%%%%%%%%%%%%%%%%%%%%%%%%%%%%%%%%%%%%%%%%%%%%%%%%%

\section{Basics}

We recall Legendre's formula
\begin{equation}\label{legendre}
  \nu_p(N) = \sum_{j=1}^\infty \lfloor \frac{N}{p^j} \rfloor = \frac{N - s_p(N)}{p-1}.
\end{equation}

\section{Criteria for factorization}

Suppose we are trying to factorize $N!$ into factors of size at least $t$.  A candidate tuple $\vec b = (b_1,\dots,b_{N'})$ is said to be \emph{admissible} if $b_j \geq t$ for all $j=1,\dots,N'$.  The \emph{undershoot} $S^-_p(\vec b)$ and \emph{overshoot} of a tuple at a prime $p$ are defined by the formulae
\begin{align*}
   S^-_p(\vec b) &\coloneqq (\nu_p(N) - \sum_{i=1}^{N'} \nu_p(b_i))_+ \\
   S^+_p(\vec b) &\coloneqq (\sum_{i=1}^{N'} \nu_p(b_i) - \nu_p(N!))_+.
\end{align*}
By the fundamental theorem of arithmetic, we obtain a perfect factorization $N! = b_1 \dots b_{N'}$ if the undershoots and overshoots all vanish.  The \emph{excess} $E(\vec b)$ is defined to be the quantity
\begin{equation}\label{excess} 
  E(\vec b) \coloneqq \sum_{i=1}^{N'} \log \frac{b_i}{t}.
\end{equation}
This quantity is non-negative for admissible tuples.  Intuitively, the smaller the excess, the more efficient the candidate factorization.

\begin{proposition}  Let $2 \leq t \leq N$.  Suppose one can find an admissible tuple $\vec b$ obeying the inequality
\begin{equation}\label{main} 
  E(\vec b) + \sum_p (S^-_p(\vec b) \log p + S^+_p(\vec b) \log 2) \leq \log N! - N \log t
\end{equation}
as well as the side condition
\begin{equation}\label{side}
 \sum_{p>2} S^+_p(\vec b) \left\lceil \frac{\log p}{\log 2} \right\rceil < S^-_2(\vec b).
\end{equation}
Then $t(N) \geq t$.
\end{proposition}


\begin{proof}  Since there is a non-zero undershoot at $p=2$, there is no overshoot at this prime. Suppose that we have an overshoot at an odd prime $p>2$, thus one of the $b_i$ has a factor of $p$. If we replace this factor of $p$ by the larger quantity $2^{\lceil \frac{\log p}{\log 2} \rceil}$, then $S^+_p(\vec b)$ goes down by one, $E(\vec b)$ goes up by at most $\log 2$, and both sides of \eqref{side} decrease by the same amount.  Thus neither \eqref{main} nor \eqref{side} can be destroyed by this process.  Iterating, we may assume without loss of generality that there are no overshoots, thus
 $$  W(\vec b) + \sum_p (\nu_p(N!) - \sum_{i=1}^{N'} \nu_p(b_i))\log p \leq \log N! - N \log t.$$
From the fundamental theorem of arithmetic we have
$$ \sum_p \nu_p(b_i) \log p = \log b_i; \quad \sum_p \nu_p(N!) \log p = \log N!.$$
Using this and \eqref{excess}, we may rearrange the previous inequality as
$$ N \log t \leq N' \log t.$$
If we then delete all but $N$ of the terms in the tuple $(b_1,\dots,b_{N'})$, and then distribute all undershoots amongst these surviving terms arbitrarily, we obtain a factorization $N! = a_1 \dots a_N$ with all $a_i \geq t$, so that $t(N) \geq t$ as required.
\end{proof}

In view of this proposition, we no longer need to keep direct track of the number of terms in the factorization; as long as we keep the excess small, and have not too much undershoot or overshoot, taking particular care to avoid undershoot at large primes, and to keep some undershoot at $2$ for the side condition, we get a lower bound on $t(N)$.  For $t = N/3$, the right-hand side $\log N! - N \log t$ is roughly $N \log \frac{3}{e} \approx 0.098 N$ by the Stirling approximation.






\section{Explicit estimates for primes}

\begin{theorem}[Buthe's bounds]\cite{buthe}  For any $2 \leq x \leq 10^{19}$, we have
$$ \mathrm{li}(x) - \frac{\sqrt{x}}{\log x}(1.95 + \frac{3.9}{\log x} + \frac{19.5}{\log^2 x}) \leq \pi(x) < \mathrm{li}(x)$$
and
$$ \mathrm{li}(x) - \frac{\sqrt{x}}{\log x} \leq \pi^*(x) < \mathrm{li}(x) + \sqrt{x}{\log x}.$$
\end{theorem}

For $x > 10^{19}$ we have the bounds of Dusart \cite{dusart}.  One such bound is
$$ |\psi(x) - x| \leq 59.18 \frac{x}{\log^4 x}.$$



\begin{thebibliography}{10}

\bibitem{buthe}
J. B\"uthe, \emph{Estimating $\pi(x)$ and related functions under partial RH assumptions}, Math. Comp., 85(301), 2483--2498, Jan. 2016.

\bibitem{dusart}
P. Dusart, \emph{Explicit estimates of some functions over primes}, Ramanujan J. \textbf{45} (2018) 227--251.


\end{thebibliography}


\end{document}
