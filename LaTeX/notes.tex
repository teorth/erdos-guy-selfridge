%\pdfoutput=1
\documentclass[12pt,a4paper,reqno]{amsart}
\newcommand\hmmax{0}
\newcommand\bmmax{0}
\usepackage{amssymb}
\usepackage{amscd}
\usepackage[pdftex,pdfpagelabels]{hyperref}
\usepackage{enumerate}
\usepackage{comment}
%\usepackage{psfig}
\usepackage{graphicx}
\usepackage{cleveref}
\usepackage{siunitx}
\usepackage{tikz-cd}
\usepackage{stix}
\usepackage{bm}
\DeclareMathAlphabet\mathbfcal{LS2}{stixcal}{b}{n}
\numberwithin{equation}{section}

%\usepackage{mathabx}


\usepackage{mathtools}%                  http://www.ctan.org/pkg/mathtools
\usepackage[tableposition=top]{caption}% http://www.ctan.org/pkg/caption
\usepackage{booktabs,dcolumn}%           http://www.ctan.org/pkg/dcolumn + http://www.ctan.org/pkg/booktabs

% Lighter notation.
%\newcommand*\mc[1]{\multicolumn{1}{c}{#1}}
%\newcommand*\tupref[2]{\href{http://math.mit.edu/~primegaps/tuples/admissible_#1_#2.txt}{\num{#2}}}



%\DeclareMathOperator*\Kl{Kl} (commented because yield bad display for  \Kl_q %replaced with \newcommand... )
%\DeclareMathOperator*\FT{FT} (commented because yield bad display for \FT_q
%replaced with \newcommand...)

\DeclareMathOperator*\swan{swan}
\DeclareMathOperator*\cond{cond}
\DeclareMathOperator*\Gal{Gal}

\newcommand{\FT}{\mathrm{FT}}
\newcommand{\Kl}{\mathcal{K}\ell}
% Setup for ``caption''.
%\DeclareCaptionLabelSeparator{separation}{:\quad}
%\captionsetup{
  %font=small,
  %labelfont=sc,
  %labelsep=separation,
  %width=0.8\textwidth
%}

\DeclareFontFamily{OT1}{rsfs}{}
\DeclareFontShape{OT1}{rsfs}{n}{it}{<-> rsfs10}{}
\DeclareMathAlphabet{\mathscr}{OT1}{rsfs}{n}{it}

\addtolength{\textwidth}{3 truecm}
\addtolength{\textheight}{1 truecm}
\setlength{\voffset}{-.6 truecm}
\setlength{\hoffset}{-1.3 truecm}

\theoremstyle{plain}

\newtheorem{theorem}{Theorem}[section]
%\newtheorem{theorem}[theorem]{Theorem}
\newtheorem{proposition}[theorem]{Proposition}
\newtheorem{lemma}[theorem]{Lemma}
\newtheorem{corollary}[theorem]{Corollary}
\newtheorem{conjecture}[theorem]{Conjecture}
\newtheorem{heuristic}[theorem]{Heuristic}
\newtheorem{principle}[theorem]{Principle}
\newtheorem{question}[theorem]{Question}
\newtheorem{problem}[theorem]{Problem}
\newtheorem{claim}[theorem]{Claim}

\theoremstyle{definition}

%\newtheorem{roughdef}[subsection]{Rough Definition}
\newtheorem{definition}[theorem]{Definition}
\newtheorem{remark}[theorem]{Remark}
\newtheorem{remarks}[theorem]{Remarks}
\newtheorem{example}[theorem]{Example}
\newtheorem{examples}[theorem]{Examples}
%\newtheorem{problem}[subsection]{Problem}
%\newtheorem{question}[subsection]{Question}

\renewcommand\P{\mathbb{P}}
\newcommand\E{\mathbb{E}}
\newcommand\Var{\mathrm{Var}}
\newcommand\R{\mathbb{R}}
\newcommand\Z{\mathbb{Z}}
\newcommand\F{\mathbf{F}}
\newcommand\N{\mathbb{N}}
\newcommand\n{\mathbf{n}}
\renewcommand\a{\mathbf{a}}
\renewcommand\b{\mathbf{b}}
\renewcommand\j{\mathbf{j}}
\renewcommand\k{\mathbf{k}}
\renewcommand\v{\mathbf{v}}
\renewcommand\t{\mathbf{t}}
\renewcommand\r{\mathbf{r}}
\renewcommand\l{\mathbf{l}}
\newcommand\X{\mathbf{X}}
\newcommand\T{\mathbf{T}}
\newcommand\Y{\mathbf{Y}}
\newcommand\A{\mathbf{A}}
\newcommand\W{\mathbf{W}}
\newcommand\C{\mathbb{C}}
\newcommand\Q{\mathbb{Q}}
\renewcommand\Re{{\operatorname{Re}}}
\renewcommand\Im{{\operatorname{Im}}}
\newcommand\Log{{\operatorname{Log}}}
\newcommand\lcm{{\operatorname{lcm}}}
\renewcommand\gcd{{\operatorname{gcd}}}
\newcommand\eps{\varepsilon}
\newcommand\tuple{{\mathcal B}}

\renewcommand{\mod}{\bmod}

\parindent 0mm
\parskip   5mm


\begin{document}

\title{Notes on upper and lower bounding $t(N)$}

\author{Terence Tao}
\maketitle

%%%%%%%%%%%%%%%%%%%%%%%%%%%%%%%%%%%%%%%%%%%%%%%%%

\section{Basics}

Let $\tuple = \{ b_1,\dots,b_M\}$ be a finite multiset of natural numbers (thus each natural number may appear in $\tuple$ multiple times); the ordering of elements in the multiset will not be of relevance to us.  The \emph{cardinality} $|\tuple| = M$ of the multiset is the number of elements counting multiplicity; for example,
$$ |\{ 2,2,3\}| = 3.$$
The \emph{product} $\prod \tuple$ of the finite multiset is defined by $\prod \tuple \coloneqq \prod_{b \in \tuple} b$, where we count for multiplicity; for example
$$ \prod \{2,2,3\} = 12.$$

The tuple $\tuple$ is a \emph{factorization} of a natural number $M$ if $\tuple = M$, and a \emph{subfactorization} if $\tuple | M$.  For example, $\{2,2,3\}$ is a factorization of $12$ and a subfactorization of $24$.

We use $\nu_p(a/b) = \nu_p(a)-\nu_p(b)$ to denote the $p$-adic valuation of a positive natural number $a/b$, that is to say the number of times $p$ divides the numerator $a$, minus the number of times $p$ divides the denominator $b$.  For instance, $\nu_2(32/27)=5$ and $\nu_3(32/27)=-3$.  By the fundamental theorem of arithmetic, we see that a finite multiset $\tuple$ is a factorization of $M$ if and only if
$$ \nu_p( M / \prod \tuple ) = 0$$
for all primes $p$, and a subfactorization if and only if
$$ \nu_p( M / \prod \tuple ) \geq 0$$
for all primes $p$.  We refer to $\nu_p( M / \tuple )$ as the \emph{$p$-surplus} of $\tuple$ (as an attempted factorization) of $M$ at prime $p$, and $-\nu_p(M/\prod\tuple) = \nu_p(\prod\tuple/M)$ as the \emph{$p$-deficit}, and say that the factorization is $p$-balanced if $\nu_p(M/\prod \tuple)=0$.  Thus a subfactorization (resp. factorization) occurs when one has non-negative surpluses (resp. balance) at all primes $p$.

\begin{example} Suppose one wishes to factorize $5! = 2^3 \times 3 \times 5$.  The attempted factorization ${\mathcal B} \coloneqq \{3,4,5,5\}$ has a $2$-surplus of $\nu_2(5!/\prod {\mathcal B}) = 1$, is in balance at $3$, and has a $5$-deficit of $\nu_2(\prod {\mathcal B}/5!) = 1$, so it is not a factorization or subfactorization of $5!$.  However, if one replaces one of the copies of $5$ in ${\mathcal B}$ with a $2$, this erases both the $2$-surplus and the $5$-deficit, and creates a factorization $\{2,3,4,5\}$ of $5!$.
\end{example}

If one applies a logarithm to the fundamental theorem of arithmetic, one obtains the identity
\begin{equation}\label{ftoa}
  \sum_p \nu_p(r) \log p = \log r
\end{equation}
for any positive rational $r$.

A finite multiset $\tuple$ is said to be \emph{$t$-admissible} for some $t>0$ if $b \geq t$ for all $b \in \tuple$.  We define $t(N)$ denotes the largest quantity such that there exists a $t(N)$-admissible factorization of $N!$ of cardinality $N$.  Clearly, $t(N)$ is also the largest quantity such that there exists a $t(N)$-admissible subfactorization of $N!$ of cardinality at least $N$, since when starting from such a subfactorization, we may delete elements and then distribute any surpluses at any primes arbitrarily to create a factorization of cardinality exactly $N$.

\begin{example}  The finite multiset $\{2,2,3,3,4,4,5,7\}$ is a $2$-admissible subfactorization of $8! = 2^7 \times 3^2 \times 5 \times 7$, having a $2$-surplus of $1$. If one deletes a copy of $2$ to make the cardinality exactly $8$, one now has a surplus of $2$ at $2$; one can distribute these two powers of $2$ to the remaining element of $2$ to obtain a factorization $\{ 3,3,4,4,5,7,8\}$ that is still $2$-admissible, and is in fact now $3$-admissible.
\end{example}

A useful measure of the efficiency of a $t$-admissible finite multiset $\tuple$ is the \emph{$t$-excess}
$$ E_t(\tuple) \coloneqq \sum_{i=1}^{N'} \log \frac{b_i}{t}
= \log \prod \tuple - |\tuple| \log t.$$

\begin{example}  The $3$-excess of $\{ 3,3,4,4,5,7,8\}$ is
$$ E_t(\{3,3,4,4,5,7,8\}) = 2 \log \frac{4}{3}  + \log \frac{5}{3} + \log \frac{7}{3} + \log \frac{8}{3} = 2.914\dots.$$
\end{example}

The $t$-excess clearly non-negative when $\tuple$ is $t$-admissible.
Combining this with \eqref{ftoa}, we obtain the basic \emph{balance identity}
\begin{equation}\label{excess-identity}
E_t(\tuple) + \sum_p \nu_p(N!/\prod\tuple) \log p = \log N! - |\tuple| \log t.
\end{equation}
In particular, when one has a subfactorization, the gap between $\log N!$ and $|\tuple| \log t$ must be somehow distributed between the $t$-excess $E_t(\tuple)$ and the $p$-surpluses $\nu_p(N!/\prod\tuple)$.  

\begin{example}  The $3$-admissible finite multiset $\{ 3,3,4,4,5,7,8\}$ is a factorization of $8!$ of cardinality $7$, and the gap
$$ \log 8! - 7 \log 3 = 2.914\dots$$
is entirely absorbed by the $3$-excess of the multiset.
If one replaces the element $8$ of this multiset with $4$, this reduces the excess to
$$ E_t(\{3,3,4,4,5,7,8\}) = 3 \log \frac{4}{3}  + \log \frac{5}{3} + \log \frac{7}{3} = 2.221\dots,$$
but creates a $2$-surplus of $1$ that contributes $\log 2 = 0.693\dots$ to \eqref{excess-identity}, restoring balance.
\end{example}

From \eqref{excess-identity}, we have the following equivalent definition of $t(N)$:

\begin{lemma}[Equivalent form of $t(N)$]\label{balance} $t(N)$ is the supremum of all $t$ for which there exists a $t$-admissible subfactorization $\tuple$ of $N!$ with
$$  E_t(\tuple) + \sum_p \nu_p(N!/\tuple) \log p \leq \log N! - N \log t.$$
\end{lemma}

The advantage of this formulation is that one no longer needs to directly track the cardinality $|\tuple|$ of the $t$-admissible subfactorization $\tuple$.  The formulation  highlights the need to locate subfactorizations in which both the $t$-excess and the $p$-surpluses are kept as low as possible.


We recall Legendre's formula
\begin{equation}\label{legendre}
  \nu_p(N!) = \sum_{j=1}^\infty \left\lfloor \frac{N}{p^j} \right\rfloor = \frac{N - s_p(N)}{p-1}.
\end{equation}

Asymptotically, it is known that
$$ \frac{1}{e} - \frac{O(1)}{\log N} \leq \frac{t(N)}{N} \leq \frac{1}{e} - \frac{c_0+o(1)}{\log N}$$
where
  \begin{align*}
    c_0 &\coloneqq \frac{1}{e} \int_0^1 \left \lfloor \frac{1}{x} \right\rfloor \log \left( ex \left \lceil \frac{1}{ex} \right\rceil \right)\ dx \\
    &= \frac{1}{e} \int_1^\infty \lfloor y \rfloor \log \frac{\lceil y/e \rceil}{y/e}\ \frac{dy}{y^2} \\
    &= 0.3044\dots
  \end{align*}

To bound the factorial, we have the explicit Stirling approximation \cite{robbins}
\begin{equation}\label{stirling}
  N \log N - N + \log \sqrt{2\pi N} + \frac{1}{12N+1} \leq \log N! \leq N \log N - N + \log \sqrt{2\pi N} + \frac{1}{12N},
\end{equation}
valid for all natural numbers $N$. 

To estimate the prime counting function, we have the following good asymptotics up to a large height.

  \begin{theorem}[Buthe's bounds]\cite{buthe}  For any $2 \leq x \leq 10^{19}$, we have
  $$ \mathrm{li}(x) - \frac{\sqrt{x}}{\log x}\left(1.95 + \frac{3.9}{\log x} + \frac{19.5}{\log^2 x}\right) \leq \pi(x) < \mathrm{li}(x)$$
  and
  $$ \mathrm{li}(x) - \frac{\sqrt{x}}{\log x} \leq \pi^*(x) < \mathrm{li}(x) + \frac{\sqrt{x}}{{\log x}}.$$
  \end{theorem}
  
  For $x > 10^{19}$ we have the bounds of Dusart \cite{dusart}.  One such bound is
  $$ |\psi(x) - x| \leq 59.18 \frac{x}{\log^4 x}.$$

  
\section{Criteria for upper bounding \texorpdfstring{$t(N)$}{t(N)}}

We have the trivial upper bound $t(N) \leq (N!)^{1/N}$.  This can be improved to $t(N) \leq N/e$ for $N \neq 1,2,4$, answering a conjecture of Guy and Selfridge \cite{guy-selfridge}; see \cite{tao}.  This was derived from the following slightly stronger criterion, which asymptotically gives $\frac{t(N)}{N} \leq \frac{1}{e} - \frac{c_0+o(1)}{\log N}$:

\begin{lemma}[Upper bound criterion]\label{upper-crit}  \cite[Lemma 2.1]{tao} Suppose that $1 \leq t \leq N$ are such that
  \begin{equation}\label{contra}
     \sum_{p > \frac{t}{\lfloor\sqrt{t}\rfloor}} \left\lfloor \frac{N}{p} \right\rfloor \log \left( \frac{p}{t} \left\lceil \frac{t}{p} \right\rceil \right) > \log N! - N \log t
  \end{equation}
  Then $t(N) < t$.
  \end{lemma}

A surprisingly sharp upper bound comes from linear programming.

\begin{lemma}[Linear programming bound]\label{lp-upper}  Let $N$ be an natural number and $1 \leq t \leq N/2$.  Suppose for each prime $p \leq N$, one has a non-negative real number $w_p$ which is weakly non-decreasing in $p$ (thus $w_p \leq w_{p'}$ when $p \leq p'$), and such that
\begin{equation}\label{pj}
 \sum_p w_p \nu_p(j) \geq 1
\end{equation}
for all $t \leq j \leq N$, and such that
\begin{equation}\label{hyp}
\sum_p w_p \nu_p(N!) < N.
\end{equation}
Then $t(N) < t$.
\end{lemma}

\begin{proof}
We first observe that the bound \eqref{pj} in fact holds for all $j \geq t$, not just for $t \leq j \leq N$.  Indeed, if this were not the case, consider the first $j \geq t$ where \eqref{pj} fails.  Take a prime $p$ dividing $j$ and replace it by a prime inthe interval $[p/2,p)$ which exists by Bertrand's postulate (or remove $p$ entirely, if $p=2$); this creates a new $j'$ in $[j/2,j)$ which is still at least $t$.  By the weakly decerasing hypothesis on $w_p$, we have
$$ \sum_p w_p \nu_p(j) \geq \sum_p w_p \nu_p(j')$$
and hence by the minimality of $j$ we have
$$ \sum_p w_p \nu_p(j) > 1, $$
a contradiction.

Now suppose for contradiction that $t(N) \geq t$, thus we have a factorization $N! = \prod_{j \geq t} j^{m_j}$ for some natural numbers $m_j$ summing to $N$.  Taking $p$-valuations, we conclude that
$$ \sum_{j \geq t} m_j \nu_p(j) \leq \nu_p(N!)$$
for all $p \leq N$.  Multiplying by $w_p$ and summing, we conclude from \eqref{pj} that
$$ N = \sum_{j \geq t} m_j \leq \sum_p w_p \nu_p(N!),$$
contradicting \eqref{hyp}.
\end{proof}


\section{Powers of 2 and 3}

We now begin the study of constructions that can establish lower bounds of the form $t(N) \geq t$ for some 
\begin{equation}\label{tn}
  1 \leq t \leq N.  
\end{equation}
It will be convenient to parameterize
\begin{equation}\label{tparam}
  t = \frac{N}{e^{1+\delta}}
\end{equation}
where we shall assume
\begin{equation}\label{delta-pos}
\delta > 0;
\end{equation}
for instance, if $t=N/3$, we will have $\delta = \log \frac{3}{3} \approx 0.098$.  We also a parameter $L \geq 1$ for which
\begin{equation}\label{tl}
  9L \leq t
\end{equation}
and divide the primes into three categories:
\begin{itemize}
\item The \emph{tiny primes} $p=2,3$;
\item The \emph{small primes} $3 < p \leq \sqrt{t/L}$;
\item The \emph{large primes} $p > \sqrt{t}/L$.
\end{itemize}

For any $B \geq 1$, define a \emph{$B$-smooth number} to be a number whose prime factors are all at most $B$.  Here we will be primarily interested in the cases $B=2,3$.

A $2$-smooth number is just a power of two; and for any $t \geq 1$, there exists a $2$-smooth number $2^n$ in the interval $[t,2t]$;
indeed one can take $n = \lfloor \log t / \log 2 \rfloor$.  For $3$-smooth numbers $2^n 3^m$ - that is to say, products of tiny primes - one can do better.  For any $L \geq 1$, let $\kappa_L$ be the least quantity such that for any real nmber $t \geq L$, there exists a $3$-smooth number $2^n 3^m$ such that
$$ t \leq 2^n 3^m \leq \exp(\kappa_L) t.$$
Thus for instance $\kappa_1 = \log 2$ thanks to the aforementioned fact about $2$-smooth numbers, and it is clear that $\kappa_L$ is non-decreasing in $L$.  We have the following explicit bounds that noticeably improve upon $\log 2 = 0.69314\dots$:

\begin{lemma}\label{lemcount-0}  If $n_1,n_2,m_1,m_2$ are positive integers such that
$$ 1 \leq \frac{3^{m_1}}{2^{n_1}}, \frac{2^{n_2}}{3^{m_2}}$$
then
$$ \kappa_{\min( 2^{n_1+n_2},3^{m_1+m_2})/6} \leq \log \max\left(\frac{3^{m_1}}{2^{n_1}}, \frac{2^{n_2}}{3^{m_2}}\right).$$
Thus, for instance, setting $n_1=3$, $m_1=2$, $n_2=2$, $m_2=1$, we have
$$ \kappa_{4.5} \leq \log \frac{2^2}{3} = 0.28768\dots,$$
setting $n_1 = 3$, $m_1 = 2$, $n_2 = 5$, $m_2 = 3$, we have
$$\kappa_{40.5} \leq \log \frac{2^5}{3^3} = 0.16989\dots$$ 
and setting $n_1 = 11$, $m_1 = 7$, $n_2 = 8$, $m_2 = 5$, we have 
$$\kappa_{2^{18}/3} \leq \log \frac{3^7}{2^{11}} = 0.06566\dots$$
($2^{18}/3 = 87381.33\dots$).
\end{lemma}

\begin{proof}  If $\min( 2^{n_1+n_2},3^{m_1+m_2})/6 \leq t \leq 2^{n_2-1} 3^{m_1-1}$, then we have
\begin{equation}\label{tb} 
  t \leq 2^{n_2-1} 3^{m_1-1} \leq \max(\frac{3^{m_1}}{2^{n_1}}, \frac{2^{n_2}}{3^{m_2}}) t,
\end{equation}
so we are done in this case.  Now suppose that $t > 2^{n_2-1} 3^{m_1-1}$.
Let $2^n 3^m$ be the smallest $3$-smooth number that is at least $t$, then we must have $n \geq n_2$ or $m \geq m_1$ (or both).  Thus at least one of $\frac{2^{n_1}}{3^{m_1}} 2^n 3^m$ and $\frac{3^{m_2}}{3^{n_2}} 2^n 3^m$ is an integer, and is thus at most $t$ by construction.  This gives \eqref{tb}, and the claim follows.
\end{proof}

Asymptotically, we have logarithmic-type decay:

\begin{lemma}[Baker bound]\label{baker} We have
  $$ \kappa_L \ll \log^{-c} L$$
for all $L \geq 2$ and some absolute constant $c>0$.
\end{lemma}

\begin{proof}  From the classical theory of continued fractions, we can find rational approximants
\begin{equation}\label{abn}
 \frac{p_{2j}}{q_{2j}} \leq \frac{\log 3}{\log 2} \leq \frac{p_{2j+1}}{q_{2j+1}}
\end{equation}
to the irrational number $\log 3/\log 2$, where the convergents $p_j/q_j$ obey the recursions
$$ p_j = b_j p_{j-1} + p_{j-2}; \quad q_j = b_j q_{j-1} + q_{j-2}$$
with $p_{-1} = 1, q={-1}=0, p_0 = b_0, q_0=1$, and $[b_0;b_1,b_2,\dots] = [1;1,1,2,\dots]$ is the continued fraction expansion of $\frac{\log 3}{\log 2}$.  Furthermore, $p_{2j+1}q_{2j} - p_{2j} q_{2j+1} = 1$, and hence
\begin{equation}\label{abn-2} 
  \frac{\log 3}{\log 2} - \frac{p_{2j}}{q_{2j}} = \frac{1}{q_{2j} q_{2j+1}}.
\end{equation}
By Baker's theorem, $\frac{\log 3}{\log 2}$ is a Diophantine number, giving a bound of the form
\begin{equation}\label{bj1}
   q_{2j+1} \ll q_{2j}^{O(1)}
\end{equation}
and a similar argument gives
\begin{equation}\label{bj2}
 q_{2j+2} \ll q_{2j+1}^{O(1)}.
\end{equation}
We can rewrite \eqref{abn} as
$$ 1 \leq \frac{3^{q_{2j}}}{2^{p_{2j}}}, \frac{2^{p_{2j+1}}}{3^{q_{2j+1}}}$$
and routine Taylor expansion using \eqref{ab-2} gives the upper bounds
$$ \frac{3^{q_{2j}}}{2^{p_{2j}}}, \frac{2^{p_{2j+1}}}{3^{q_{2j+1}}}\leq \exp\left( O\left( \frac{1}{q_{2j}}\right)\right).$$
If one then sets $j$ to be the largest natural number for which \eqref{atb} holds with $t$ replaced by $L$, the claim then follows from \eqref{bj1}, \eqref{bj2}, and \Cref{lemcount-0}.
\end{proof}


We can now obtain efficient $t$-admissible subfactorizations of $2^n 3^m$ when $n,m$ are somewhat comparable.

\begin{lemma}\label{bound23}  Let $L \geq 1$.  Let $t > 3L$ and let $2^n 3^m$ be a $3$-smooth number obeying the conditions
\begin{equation}\label{B-bound}
\frac{\log(3L)+\kappa}{\log t - \log(3L)} \leq \frac{n \log 2}{m \log 3} \leq \frac{\log t - \log(2L)}{\log(2L)+\kappa}.
\end{equation}
Then one can find a $t$-admissible subfactorization $\tuple$ of $2^n 3^m$ such that
\begin{equation}\label{excess-bound} 
  E_t(\tuple) \leq \frac{\kappa_L}{\log t} (n \log 2 + m \log 3)
\end{equation}
and
\begin{equation}\label{surplus-bound} 
  |\nu_2(2^n 3^m/\tuple)|_{\log 2,\infty} + |\nu_3(2^n 3^m/\tuple)|_{\log 3,\infty}  \leq 2(\log t + \kappa_L).
\end{equation}
\end{lemma}

\begin{proof}  Let $2^{n_0}, 3^{m_0}$ be the largest powers of $2$ and $3$ less than $t/L$ respectively.  By definition of $\kappa_L$, we can find $3$-smooth numbers $2^{n_1} 3^{m_1}$, $2^{n_2} 3^{m_2}$  such that
\begin{equation}\label{ab} 
  \frac{t}{2^{n_0}} \leq 2^{n_1} 3^{m_1} \leq e^{\kappa} \frac{t}{2^{n_0}}
\end{equation}
and
\begin{equation}\label{ab-2}
 \frac{t}{3^{m_0}} \leq 2^{n_2} 3^{m_2} \leq e^{\kappa} \frac{t}{3^{m_0}},
\end{equation}
or equivalently
\begin{equation}\label{ab-equiv} 
  t \leq 2^{n_0+n_1} 3^{m_1}, 2^{n_2} 3^{m_0+m_2} \leq e^{\kappa} t.
\end{equation}
We can bound
\begin{align*}
  \frac{n_0 + n_1}{m_1} &\geq \frac{n_0}{\log (e^{\kappa} \frac{t}{2^{n_0}}) / \log 3} \\
  &\geq \frac{(\log t - \log(2L)) / \log 2}{ (\log(3L)+\kappa) / \log 3 }
\end{align*}
(with the convention that this bound is vacuously true for $m_1=0$) and similarly
\begin{align*}
  \frac{n_2}{m_0+m_2} &\leq \frac{\log(e^\kappa \frac{t}{3^{m_0}}) / \log 2}{m_0} \\
  &\leq \frac{(\log(2L)+\kappa)/\log 2}{(\log t-\log(3L))/\log 3}
\end{align*}
and hence by \eqref{B-bound}
\begin{equation}\label{m-wedge} \frac{n_2}{m_0+m_2} \leq \frac{n}{m} \leq \frac{n_0+n_1}{m_1}.
\end{equation}
Thus we can write $(n,m)$ as a non-negative linear combination
$$ (n,m) = \alpha_1 (n_0+n_1,m_1) + \alpha_2 (n_2,m_0+m_2)$$
for some real $\alpha_1, \alpha_2 \geq 0$. We now take our subfactorization $\tuple$ to consist of $\lfloor \alpha_1$ copies of the $3$-smooth number $2^{n_0+n_1} 3^{m_1}$ and $\lfloor \alpha_2 \rfloor$ copies of the $3$-smooth number $2^{n_2} 3^{m_0+m_2}$.  By \eqref{ab-equiv}, each term $2^{n'} 3^{m'}$ here is admissible and contributes an excess of at most $\kappa$, which is in turn bounded by $\frac{\kappa}{\log t} (n' \log 2 + m' \log 3)$.  Adding these bounds together, we obtain \eqref{excess-bound}.

The expression $2^n 3^m / \prod \tuple$ contains at most $n_0+n_1+n_2$ factors of $2$ and at most $m_0+m_2+m_1$ factors of $3$, hence
$$ \nu_2(2^n 3^m / \prod \tuple) \log 2 +
\nu_3(2^n 3^m / \prod \tuple) \log 3
\leq \log 2^{n_0+n_1} 3^{m_1} + \log 2^{n_2} 3^{m_0+m_2},$$
and the bound \eqref{surplus-bound} follows.
\end{proof}



\section{Criteria for lower bounding \texorpdfstring{$t(N)$}{t(N)}}

\Cref{balance} gives an initial criterion for lower bounding $t(N)$.  We now perform various manipulations on tuples to replace this criterion with a more tractable one.
For $a_+,a_- \in [0,+\infty]$, we define the asymmetric norm $|x|_{a_+,a_-}$ of a real number $x$ by the formula
$$ |x|_{a_+,a_-} \coloneqq \max(a_+ x, -a_- x),$$
thus this is $a_+ |x|$ when $x$ is positive and $a_- |x|$ when $x$ is negative.  If $a_+,a_-$ are finite, this function is Lipschitz with constant $\max(a_+,a_-)$.  One can think of $a_+$ as the ``cost'' of making $x$ positive, and $a_-$ as the
``cost'' of making $x$ negative.  One can then reformulate \Cref{balance} as follows.

\begin{proposition}[Reformulated balance criterion]\label{balance-reform}  Let $1 \leq t \leq N$, and suppose that one has a $t$-admissible tuple $\tuple$ obeying the following hypothesis:
  \begin{itemize}
  \item[(i)] (Small excess and surplus at all primes)
  \begin{equation}\label{new-balance}
    E_t(\tuple) + \sum_p |\nu_p(N!/\prod \tuple)|_{\log p,\infty} \leq \log N! - N \log t.
\end{equation}
\end{itemize}
Then $t(N) \geq t$.
\end{proposition}

Indeed, the infinite penalty for making $\nu_p(N!/\tuple)$ in \eqref{new-balance} ensures that $\tuple$ is a subfactorization of $N!$.

We will reduce this infinite penalty term later, but let us work on other aspects of the criterion \eqref{new-balance} first.  In practice we will apply this criterion with $t \coloneqq N / e^{1+\delta}$ for some $\delta>0$; for instance, if we wish to set $t = N/3$, then $\delta = \log \frac{e}{3} \approx 0.098$.  From \eqref{stirling} we may then replace $\log N! - N \log t = \log N! - N \log N + N + \delta N$ by the slightly smaller quantity
$$ \delta N + \log \sqrt{2\pi N}.$$
The $\log \sqrt{2\pi N}$ is a lower order term, and we shall use it only to clean up some other lower order terms.

Using \Cref{bound23}, we can leave a large surplus at tiny primes and still get good bounds:

\begin{proposition}[Criterion with tiny-prime surplus]\label{balance-23}  Let $L \geq 1$.
  Let $3L < t = N/e^{1+\delta}$ for some $\delta>0$, and suppose that one has a $t$-admissible tuple $\tuple$ obeying the following hypotheses:
\begin{itemize}
  \item[(i)] (Small excess and surplus at non-tiny primes)
  \begin{equation}\label{new-balance-3}
    E_t(\tuple) + \sum_{p>3} |\nu_p(N!/\prod \tuple)|_{\log p,\infty} \leq \delta N + \kappa_L - \frac{3}{2} \log N - \kappa_L (\log \sqrt{12}) \frac{N}{\log t}.
  \end{equation}
  \item[(ii)] (Large surpluses at tiny primes)
The surpluses $\nu_2(N!/\prod \tuple), \nu_3(N!/\prod \tuple)$ are positive and obey the bounds
$$
\frac{\log(3L)+\kappa_L}{\log t - \log(3L)} \leq \frac{\nu_2(N!/\prod \tuple) \log 2}{\nu_3(N!/\prod \tuple) \log 3} \leq \frac{\log t - \log(2L)}{\log(2L)+\kappa_L}.
$$
\end{itemize}
Then $t(N) \geq t$.
\end{proposition}
  

\begin{proof} Write $n \coloneqq \nu_2(N!/\prod \tuple)$ and $m \coloneqq \nu_3(N!/\prod \tuple)$.  From \eqref{legendre} we have $n \leq N$ and $m \leq N/2$, hence
  $$ n \log 2 + m \log 3 \leq N \log \sqrt{12}.$$
Applying \Cref{bound23}, we can find a subfactorization $\tuple'$ of $2^n 3^m$ with an excess of at most $(\kappa_L \log \sqrt{12}) \frac{N}{\log t}$, and with
$$ |\nu_2(2^n 3^m/\prod \tuple')|_{\log 2,\infty}
+ |\nu_3(2^n 3^m/\prod \tuple')|_{\log 3,\infty}
 \leq 2(\log t + \kappa_L) \leq 2 \log N - 2 + 2 \kappa_L.$$
 If we let $\tuple''$ be the concatenation of $\tuple$ and $\tuple'$, then $\tuple''$ is another $t$-admissible tuple, and from \eqref{new-balance-3} and the observation that $-2+3\kappa_L \leq \log \sqrt{2\pi}$, we see that
 $$
 E_t(\tuple'') + \sum_{p} |\nu_p(N!/\prod \tuple'')|_{\log p,\infty} \leq \delta N + \log \sqrt{2\pi N},$$
 and the claim now follows from \Cref{balance-reform}.
\end{proof}

The criterion \eqref{new-balance-3} will still be somewhat expensive at small primes $3 < p \leq \sqrt{t/L}$.  We can improve the situation as follows.

\begin{proposition}[Improved criterion with tiny-prime surplus]\label{balance-23'}  Let $L \geq 1$.
  Let $9L < t = N/e^{1+\delta}$ for some $\delta>0$, and suppose that one has a $t$-admissible tuple $\tuple$ obeying the following hypotheses.
\begin{itemize}
\item[(i)] (Small excess and surplus at non-tiny primes)
\begin{equation}\label{new-balance-4}
  \begin{split}
&      E_t(\tuple) + \sum_{3 < p \leq \sqrt{t/L}}
 |\nu_p(N!/\prod \tuple)|_{\frac{\kappa_L \log p}{\log \sqrt{t/L}},\infty} + 
 \sum_{p>\sqrt{t/L}} |\nu_p(N!/\prod \tuple)|_{\log p,\infty}\\
&\quad    \leq \delta N - \frac{3}{2} \log N - \kappa_L (\log \sqrt{12}) \frac{N}{\log t}.
  \end{split}
 \end{equation}
 \item[(ii)] (Large surpluses at tiny primes) Whenever $n_{**}, m_{**}$ are natural numbers obeying the  bounds
 $$ n_{**} \log 2 + m_{**} \log 3 \leq 
 \sum_{3 < p \leq \sqrt{t/L}}
 |\nu_p(N!/\prod \tuple)|_{\frac{\log \sqrt{tL} + \kappa}{\log\sqrt{t/L}} \log p,\infty}
 + \log t + \kappa,$$
 then $\nu_2(N!/\prod \tuple) > n_{**}$, $\nu_3(N!/\prod \tuple) > m_{**}$, and furthermore
 $$
 \frac{\log(3L)+\kappa_L}{\log t - \log(3L)} \leq \frac{(\nu_2(N!/\prod \tuple)-n_{**}) \log 2}{(\nu_3(N!/\prod \tuple)-m_{**}) \log 3} \leq \frac{\log t - \log(2L)}{\log(2L)+\kappa_L}.$$
 \end{itemize}
  Then $t(N) \geq t$.
\end{proposition}

\begin{proof} By \eqref{new-balance-4}, $\tuple$ is a subfactorization of $N!$. Consider all the $p$-surplus primes in the range $3 < p \leq \sqrt{t/L}$, thus each such prime is considered with multiplicity $\nu_p(N!/\prod \tuple)$.  Denoting their product as $B$, we have
  $$ \log B = \sum_{3 < p \leq \sqrt{t/L}}
  |\nu_p(N!/\prod \tuple)|_{\log p,\infty}.$$
Using the greedy algorithm, one can factor $B$ into $M$ factors $c_1,\dots,c_M$ in the interval $[\sqrt{t/L}, \frac{t/L}]$, times one exceptional factor $c_*$ in $[1,\sqrt{t/L}]$, for some $M$.  We have the bound
$$ (\sqrt{t/L})^M \leq B$$
and hence
$$ M \leq \sum_{3 < p \leq \sqrt{t/L}}
|\nu_p(N!/\prod \tuple)|_{\frac{\log p}{\log\sqrt{t/L}},\infty}.$$
For each of the $M$ factors $c_i$, we may use the definition of $\kappa_L$ and find $3$-smooth number $2^{n_i} 3^{m_i}$ in the interval $[t/c_i,e^{\kappa_L} t/c_i]$, and similarly for the exceptional factor $c_*$ we may find a $3$-smooth number $2^{n_*} 3^{m_*}$ in the interval $[t/c_*,e^{\kappa_L} t/c_*]$.  If we then adjoin the $3$-smooth numbers $2^{n_i} 3^{m_i} c_i$ for $i=1,\dots,M$ as well as $2^{n_*} 3^{m_*} c_*$ to the tuple $\tuple$ to create a new tuple $\tuple'$.  This tuple is still $t$-admissible, but now has no $p$-surplus (or $p$-deficit) at any prime $3 < p \leq \sqrt{t/L}$.  The quantity $n_i \log 2 + m_i \log 3$ is bounded by $\log \sqrt{tL} + \kappa_L$, and the quantity $n \log 2 + m \log 3$ is similarly bounded by $\log t + \kappa$, hence if we denote $n_{**} \coloneqq n_1 + \dots + n_M + n_*$ and $m_{**} \coloneqq m_1 + \dots + m_M + m_*$, we have
$$ n_{**} \log 2 + m_{**} \log 3 \leq 
\frac{\log \sqrt{tL} + \kappa_L}{\log\sqrt{t/L}}
\sum_{3 < p \leq \sqrt{t/L}}
|\nu_p(N!/\prod \tuple)|_{\log p,\infty}
+ \log t + \kappa_L.$$
By hypothesis, we now see that $\tuple'$ has no $p$-deficit at $2$ or $3$ either, so $\tuple'$ is still a subfactorization of $N!$.  Each of the new factors in $\tuple'$ contributes an excess of at most $\kappa_L$, so the total excess of $\tuple'$ is at most
$$ E_t(\tuple) + \kappa_L M + \kappa_L$$
which is in turn bounded by
$$ E_t(\tuple) + \sum_{3 < p \leq \sqrt{t/L}}
|\nu_p(N!/\prod \tuple)|_{\frac{\kappa_L \log p}{\log\sqrt{t/L}},\infty} + \kappa_L.$$
We conclude that $\tuple'$ obeys the hypotheses of \Cref{new-balance-3}, and the claim follows.
\end{proof}

Finally, we relax the subfactorization condition by permitting some $p$-deficit at various non-tiny primes $p>3$.

\begin{proposition}[Improved criterion with tiny-prime surplus, and some deficit]\label{balance-23''}  Let $L \geq 1$.  Let $9L < t = N/e^{1+\delta}$ for some $\delta>0$, and suppose that one has a $t$-admissible tuple $\tuple$ with the property that whenever $n_{**}, m_{**}$ are natural numbers obeying the bounds
\begin{align*}
 n_{**} \log 2 + m_{**} \log 3 &\leq 
  \sum_{3 < p \leq \sqrt{t/L}}
  |\nu_p(N!/\prod \tuple)|_{\frac{\log \sqrt{tL} + \kappa_L}{\log\sqrt{t/L}} \log p,\log p+\kappa_p}\\
&\quad   + 
  \sum_{p > \sqrt{t/L}} |\nu_p(N!/\prod \tuple)|_{0,\log p + \kappa_p}   
  + \log t + \kappa,
\end{align*}
  then $\nu_2(N!/\prod \tuple) > n_{**}$, $\nu_3(N!/\prod \tuple) > m_{**}$, and furthermore
$$
\frac{\log(3L)+\kappa_L}{\log t - \log(3L)} \leq \frac{(\nu_2(N!/\prod \tuple)-n_{**}) \log 2}{(\nu_3(N!/\prod \tuple)-m_{**}) \log 3} \leq \frac{\log t - \log(2L)}{\log(2L)+\kappa_L},
$$
and furthermore suppose that
  \begin{equation}\label{new-balance-5}
    \begin{split}
&      E_t(\tuple) + \sum_{3 < p \leq \sqrt{t/L}}
   |\nu_p(N!/\prod \tuple)|_{\frac{\kappa_L \log p}{\log \sqrt{t/L}},\kappa_p} + 
   \sum_{p>\sqrt{t/L}} |\nu_p(N!/\prod \tuple)|_{\log p,\kappa_p}\\
&\quad    \leq \delta N - \frac{3}{2} \log N - \kappa_L (\log \sqrt{12}) \frac{N}{\log t}.
    \end{split}
  \end{equation}
  Then $t(N) \geq t$.
\end{proposition}

\begin{proof} Consider all the primes with a positive deficit, that is to say the primes $p$ with a multiplicity of $|\nu_p(N!/\prod \tuple)|_{0,1}$.  If $p$ is one of these primes, we select an element of the tuple that contains $p$ as a factor, and replace it with the least $3$-smooth number $2^{n_p} 3^{n_p}$ larger than $p$, thus increasing this element by a factor of at most $\exp(\kappa_p)$; meanwhile, $\nu_2(N!/\prod \tuple) \log 2 + \nu_3(N!/\prod \tuple) \log 3$ goes down by at most $\log p + \kappa p$.  Performing this for all the primes in deficit, we can clear this deficit at the cost of raising the excess of $\tuple$ by at most $\sum_{p>3} \kappa_p$, and decreasing $\nu_2(N!/\prod \tuple)$, $\nu_3(N!/\prod \tuple)$ by some $n,m$ with $n \log 2 + m \log 3 \leq \sum_p |\nu_p(N!/\prod \tuple)|_{0,\log p+\kappa_p}$.  The hypotheses of \Cref{balance-23'} are now satisfied, and we are done.
\end{proof}
\begin{thebibliography}{10}

\bibitem{buthe}
J. B\"uthe, \emph{Estimating $\pi(x)$ and related functions under partial RH assumptions}, Math. Comp., 85(301), 2483--2498, Jan. 2016.

\bibitem{dusart}
P. Dusart, \emph{Explicit estimates of some functions over primes}, Ramanujan J. \textbf{45} (2018) 227--251.

\bibitem{guy-selfridge}
R. K. Guy, J. L. Selfridge, \emph{Factoring factorial $n$}, Amer. Math. Monthly \textbf{105} (1998) 766--767.

\bibitem{robbins}
H. Robbins, \emph{A Remark on Stirling's Formula}, Amer. Math. Monthly \textbf{62} (1955) 26--29.

\bibitem{tao}
T. Tao, \emph{Decomposing factorials into bounded factors}, preprint, 2025. \url{https://arxiv.org/abs/2503.20170}

\end{thebibliography}


\end{document}
